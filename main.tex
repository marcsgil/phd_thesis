\documentclass[a4paper,10pt]{report}
\usepackage{amsmath, amsfonts, amssymb, physics}
\usepackage{revsymb}

\title{PhD Thesis}
\author{M. Gil de Oliveira}
\date{\today}

\begin{document}

\maketitle

\chapter{Wave Optics}

\section{Quadratic Hamiltonians}

We consider hamiltonians of the form
\begin{equation}
\hat{H} = \frac{1}{2} \sum_{j,k} h_{jk} \hat{\xi}_j \hat{\xi}_k
\end{equation}
where $\boldsymbol{\hat{\xi}} = (\hat{x}_1, \ldots, \hat{x}_n, \hat{p}_1, \ldots, \hat{p}_n)$ is a vector of position and momentum operators ($\hat{p}_j = -i \lambdabar \partial_{x_j}$) satisfying the canonical commutation relations
\begin{equation}
[\hat{\xi}_j, \hat{\xi}_k] = i \lambdabar \Omega_{jk}
\end{equation}
with
\begin{equation}
\boldsymbol{\Omega} = \begin{pmatrix}0 & I_n \\ -I_n & 0 \end{pmatrix}
\end{equation}
being the symplectic form. Furthermore, $h_{jk}$ are the entries of a real symmetric matrix $\mathbf{h}$. Furthermore, $\lambdabar = \lambda/(2\pi) = 1/k$ is the reduced wavelength.

A state $\ket{\psi}$ evolves according to the Schrödinger equation
\begin{equation}
    \label{eq:schrodinger_like_equation}
i \lambdabar \frac{\partial}{\partial \eta} \ket{\psi(\eta)} = \hat{H} \ket{\psi(\eta)}
\end{equation}
where $\eta$ is an abstract evolution parameter. This has a solution
\begin{equation}
    \label{eq:schrodinger_like_solution}
    \ket{\psi(\eta)} = \hat{U}(\eta) \ket{\psi(0)}
\end{equation}
where
\begin{equation}\hat{U}_\eta = \exp\left(-\frac{i}{\lambdabar} \hat{H} \eta\right)\end{equation}
is a unitary operator.

In the Heisenberg picture, operators evolve according to
\begin{equation}
\frac{d\boldsymbol{\hat{\xi}}}{d\eta} = \frac{i}{\lambdabar} [\hat{H}, \boldsymbol{\hat{\xi}}] = \boldsymbol{\Omega} \mathbf{h} \boldsymbol{\hat{\xi}}
\end{equation}
which has the solution
\begin{equation}
\boldsymbol{\hat{\xi}}(\eta) = \mathbf{S}(\eta) \boldsymbol{\hat{\xi}}(0)
\end{equation}
where
\begin{equation}\mathbf{S}(\eta) = \exp(\boldsymbol{\Omega} \mathbf{h} \eta)\end{equation}
is a symplectic matrix, i.e., it satisfies
\begin{equation}\mathbf{S}\boldsymbol{\Omega} \mathbf{S}^T = \boldsymbol{\Omega}.\end{equation}

\section{Free Propagation}

The propagation of light within the paraxial approximation is governed by the paraxial wave equation
\begin{equation}
    2ik\partial_z u(\mathbf{r}, z) = -\partial_x^2 u(\mathbf{r}, z).
\end{equation}

This may be put in the form of equation \eqref{eq:schrodinger_like_equation} by identifying $\eta = z$, and
\begin{equation}
    \mathbf{h} = \begin{pmatrix}
        0 & 0 \\
        0 & 1
    \end{pmatrix} .
\end{equation}
A direct computation shows that $(\boldsymbol{\Omega}\mathbf{h})^2 = \boldsymbol{0}$, so that $\mathbf{S}(z)$, which, in this case, we denote by $\mathbf{FP}_z$, is given by
\begin{equation}
    \mathbf{FP}_z = \boldsymbol{I} + \boldsymbol{\Omega}\mathbf{h} z = \begin{pmatrix}
        1 & z \\
        0 & 1
    \end{pmatrix} .
\end{equation}

\section{Thin lens}

A thin lens of focal length $f$ is described can be described as imposing a phase shift $u \mapsto u e^{-ikx^2/f}$. Upon comparison with equation \eqref{eq:schrodinger_like_solution}, we see that this corresponds to the Hamiltonian
\begin{equation}
    \mathbf{h} = \begin{pmatrix}
        1 & 0 \\
        0 & 0
    \end{pmatrix}
\end{equation}
with the parameter $\eta = 1/f$. A direct computation shows that $(\boldsymbol{\Omega}\mathbf{h})^2 = \boldsymbol{0}$, so that $\mathbf{S}(1/f)$, which, in this case, we denote by $\mathbf{L}_f$, is given by
\begin{equation}
    \mathbf{L}_f = \boldsymbol{I} + \boldsymbol{\Omega}\mathbf{h} / f = \begin{pmatrix}
        1 & 0 \\
        -1/f & 1
    \end{pmatrix} .
\end{equation}

\section{Composition}

The composition of optical elements is described by the matrix product of the corresponding symplectic matrices. For example, the propagation through a thin lens of focal length $f$ followed by a free propagation of length $z$ is described by
\begin{equation}
    \mathbf{FP}_z \mathbf{L}_f = \begin{pmatrix}
        1 - z/f & z \\
        -1/f & 1
    \end{pmatrix} .
\end{equation}

If we include another free propagation of length $z_1$ before the lens, we have
\begin{equation}
    \mathbf{FP}_{z_2} \mathbf{L}_f \mathbf{FP}_{z_1} = \begin{pmatrix}
        1 - z_2/f & z_1 + z_2 - z_1 z_2 / f \\
        -1/f & 1 - z_1 / f
    \end{pmatrix} .
\end{equation}

On the other hand, if we have a lens after the free propagation, we have
\begin{equation}
    \mathbf{L}_{f_2} \mathbf{FP}_{z} \mathbf{L}_{f_1} = \begin{pmatrix}
        1 - z/f_1 & z \\
        z / (f_1 f_2) - 1/f_1 - 1/f_2 & 1 - z / f_2
    \end{pmatrix} .
\end{equation}

Finally, a telescope like system, in which two lenses of focal lengths $f_1$ and $f_2$ are separated by free space propagation, is described by
\begin{equation}
    \mathbf{FP}_{z_3} \mathbf{L}_{f_2} \mathbf{FP}_{z_2} \mathbf{L}_{f_1} \mathbf{FP}_{z_1} = \begin{pmatrix}
        (1-z_3/f_2)(1-z_2/f_1) - z_3 / f_1 & (1 - z_3 / f_2)(z_1 + z_2 - z_1 z_2 / f_1) + z_3 (1 - z_1 / f_1) \\
        z_2 / (f_1 f_2) - 1/f_1 - 1/f_2 & 1 - z_1 / f_1 + (z_1 z_2 / f_1 - z_1 - z_2)/f_2
    \end{pmatrix} .
\end{equation}

In order to obtain a proper imaging system, we would like to obtain a diagonal matrix. The lower left entry vanishes if $z_2 = f_1 + f_2$. Additionally, the upper right entry vanishes if $z_1 = f_1$ and $z_3 = f_2$. In this case, we obtain 
\begin{equation}
    \mathbf{FP}_{f_2} \mathbf{L}_{f_2} \mathbf{FP}_{f_1+f_2} \mathbf{L}_{f_1} \mathbf{FP}_{f_1} = \begin{pmatrix}
        -f_2 / f_1 & 0 \\
        0 & -f_1 / f_2
    \end{pmatrix} .
\end{equation}
We see that the image is magnified by a factor $m = -f_2 / f_1$, where the minus sign indicates that the image is inverted.

\chapter{My Publications}

I have published the following papers: \cite{PhysRevA.108.013503, DEOLIVEIRA2024110983, 48bj-bm8b}

\bibliographystyle{unsrt}
\bibliography{refs}

\end{document}